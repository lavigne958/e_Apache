\documentclass{report}

\usepackage[utf8]{inputenc}
\usepackage[T1]{fontenc}
\usepackage[francais]{babel}

\title{Rapport projet Un server HTTP}
\author{Lavigne Alexandre - Vallade Vincent}
\date{04/01/2017}


\begin{document}

\maketitle

\section*{Introduction}

Ce rapport décris notre raisonement et les solutions apportées pour répondres aux éxercices donnés
par le projet de construction d'un mini-serveur HTTP pour l'UE Programation Répartie
\newline
Ce serveur suit le protocole HTTP 1.1. par aileurs, il n'accepte que la requette de type GET (dont le détail sera décris par la suite)

Le projet est élaboré en langage C et la constante: XOPEN\_SOURCE est définit avec la valeur: 700



\section{Structure du Server}

Le serveur HTTP prend en argument à son lancement:
\begin{itemize}
\item un port d'écote
\item un nombre de client maximum à la fois
\item un nombre d'octet qu'un client peut demander au maximum par minute
\end{itemize}



\end{document}
